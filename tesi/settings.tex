% -----------------------
% Encoding and Language
% -----------------------
\usepackage[utf8]{inputenc} % libreria per accettare caratteri speciali
\usepackage[T1]{fontenc}    % libreria per la corretta gestione del font
\usepackage[italian]{babel} % libreria per scrivere in italiano
\usepackage{csquotes}
% \usepackage{showkeys}     % libreria per mostrare le etichette
\hyphenation{ }             % lista parole da tagliare nel modo giusto andando a capo

% -----------------------
% Page Layout
% -----------------------
\usepackage{geometry}
\geometry{left=2.5cm, right=2.5cm, headheight=2cm}
\linespread{1.3}
\usepackage{fancyhdr}       % libreria per impostare il documento
\renewcommand{\chaptermark}[1]{\markboth{\thechapter.\ #1}{}}
\renewcommand{\sectionmark}[1]{\markright{\thesection.\ #1}}

\fancypagestyle{fancy}{
    \fancyhf{} % Pulisce tutte le impostazioni di intestazione e piè di pagina
    \fancyhead[LE,RO]{\thepage}
    \fancyhead[RE]{\leftmark}
    \fancyhead[LO]{\rightmark}
}

% Stile per la prima pagina di ogni capitolo
%\fancypagestyle{plain}{
%  \fancyhf{}
%  \fancyfoot[C]{\thepage}
%  \renewcommand{\headrulewidth}{0pt}
%}

% Stile per capitoli speciali come dedica, introduzione, ecc.
\fancypagestyle{special}{
  \fancyhf{}
  \fancyhead[LE,RO]{\thepage}
  \fancyhead[RE,LO]{\leftmark}
  \fancyfoot[C]{}
}

% -----------------------
% Bibliography
% -----------------------
\usepackage[backend=biber]{biblatex}       % libreria per bibliografia
\addbibresource{references.bib}

% -----------------------
% Graphics
% -----------------------
\usepackage{graphicx}       % libreria per inserire grafici
\graphicspath{ {./images/} }
\usepackage{pgfplots}
\pgfplotsset{compat=newest}
\usepackage{subcaption}
\usepackage{neuralnetwork}

% -----------------------
% Tables 
% -----------------------
\usepackage{booktabs}

% -----------------------
% Links 
% -----------------------
\usepackage{hyperref}
\hypersetup{
    colorlinks=true,
    linkcolor=blue,
    urlcolor=cyan,
    pdftitle={Tesi Arcara},
    pdfpagemode=FullScreen,
}

% -----------------------
% Fonts and Text
% -----------------------
\usepackage{newlfont}       % libreria per utilizzare font particolari (\textsc{})
\usepackage{indentfirst}    % libreria per avere l'indentazione all'inizio dei capitoli, ...
\usepackage{xcolor}

% -----------------------
% Math
% -----------------------
\usepackage{amssymb, amsmath, amsthm} % librerie matematiche

% -----------------------
% Code Listings
% -----------------------
\usepackage{listings}       % libreria per snippets di codice
\definecolor{codegreen}{rgb}{0,0.6,0}
\definecolor{codegray}{rgb}{0.5,0.5,0.5}
\definecolor{codepurple}{rgb}{0.58,0,0.82}
\definecolor{backcolour}{rgb}{0.95,0.95,0.92}

\lstdefinestyle{mystyle}{
    backgroundcolor=\color{backcolour},   
    commentstyle=\color{codegreen},
    keywordstyle=\color{magenta},
    numberstyle=\tiny\color{codegray},
    stringstyle=\color{codepurple},
    basicstyle=\ttfamily\footnotesize,
    breakatwhitespace=false,        
    breaklines=true,                 
    captionpos=b,                    
    keepspaces=true,                 
    numbers=left,                    
    numbersep=5pt,                  
    showspaces=false,                
    showstringspaces=false,
    showtabs=false,                  
    tabsize=2
}

\lstset{style=mystyle}

% -----------------------
% New Commands
% -----------------------
\newcommand{\blankpage}{
    \clearpage
    \ifodd\value{page}\else
        \null
        \thispagestyle{empty}
        \newpage
    \fi
}
\newcommand{\createList}[1]{%
    #1
    \blankpage
}
\newcommand{\createChapter}[2]{%
    \chapter{#1}
    \input{chapters/#2}
    \blankpage
}
\newcommand{\createSpecialChapter}[1]{%
    \chapter*{#1}
    \markboth{#1}{}
    \addcontentsline{toc}{chapter}{#1}
    \input{chapters/#1}
    \blankpage
}
\newcommand{\confusionmatrix}[4]{%
    \begin{tabular}{lll}
        \toprule
        & Normale (previsto) & Zombie (previsto) \\
        \midrule
        Normale (reale) & #1 & #2 \\
        Zombie (reale) & #3 & #4 \\
        \bottomrule
    \end{tabular}
}
\newcommand{\scores}[7]{%
    \begin{tabular}{llll}
        \toprule
        & Normale & Zombie & All \\
        \midrule
        Precision & #1 & #2 & #3 \\
        Recall & #4 & #5 & #6 \\
        $F_1$ & & & #7 \\
        \bottomrule
    \end{tabular}
}
