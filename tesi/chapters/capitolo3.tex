\label{chap:machine_learning}
\section{Preparazione dei dati per il task di ML}
\subsection{Creazione del dataset}
\subsection{Trasformazione delle serie storiche multivariate multiple}

% dobbiamo convertire le serie storiche in un tipo di dati strutturato,
% tabellare, utilizzabile dai modelli di Machine Learning

% padding e truncate delle serie storiche 

% avg pooling

% tabular transformation

% tensor transformation

\subsection{Creazione delle feature}

% jobid e idx

% jobid e idx sono creati dal Submit Node al "concepimento" del job (es.
% 'sn-01', 'ce03-htc', ...). I S.N. sono indipendenti tra loro per cui in linea
% di principio possono esistere due job diversi con (jobid,idx) uguale (in tal
% caso vengono da S.N. diversi)%

% job work type

% job type

% one hot encoding 

% non verranno considerati i jobs con runtime <
% 1h, poichè non rilevanti per il sistema.

\subsection{Etichettatura dei dati}

% Una possibile strategia:
% - usando htjob si trovano i job eliminati per "toomuchtime": sono quelli con
%   jobstatus = 3 e runtime ~=7gg (non c'è valore esatto!)
% - a questo punto possiamo possiamo impostare un supervised learning
%   guardando su hj come hanno "vissuto" quei job.

\subsection{Tecniche di bilanciamento dei dati}

% undersample

% oversample

% class weighting in loss function

% metriche
\section{Selezioni dei modelli}
\subsection{Modelli supervisionati}
\subsection{Modelli non supervisionati}
\section{Valutazione delle performance}
\subsection{Metriche di valutazione}
\subsection{Convalida incrociata}
