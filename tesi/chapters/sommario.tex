Il CNAF gestisce un centro di calcolo dotato di oltre 46000 core distribuiti
su 960 host fisici. I job vengono accodati e schedulati dal sistema batch
(HTCondor) attraverso l'uso di algoritmi di ``fairshare''.
Durante l'esecuzione, vengono monitorate grandezze quali il consumo di memoria
e lo spazio su disco, che vengono campionate ogni tre minuti e raccolte in un
database insieme ai dati di accounting relativi ai job terminati.
Questi job possono variare notevolmente in termini di durata, da pochi minuti
a più giorni.
Questo studio esplora l'uso di tecniche di Machine Learning per prevedere il
successo o il fallimento dei job, basandosi sull'evoluzione del loro stato nel
tempo.
In particolare, ci si è concentrati sui cosiddetti ``job zombie'', ossia quei
job che, pur terminando, non rilasciano l'host fisico, causando una perdita di
risorse fino al loro timeout.
Un approccio iniziale, che prendeva in considerazione solo la prima ora di
vita del job, ha permesso di identificare con un'accuratezza del 72\% la
classe meno rappresentata (i job zombie).
Per migliorare l'accuratezza, si è preso in considerazione l'intero primo
giorno, applicando tecniche di Deep Learning sia supervisionate (CNN,
CNN+LSTM, LSTM e Transformer) che non supervisionate (autoencoder e
variational autoencoder). 
Nonostante l'incremento della complessità dei modelli, le reti neurali hanno
mostrato una tendenza all'overfitting a causa dell'estremo sbilanciamento dei
dati.
